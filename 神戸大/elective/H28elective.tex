\documentclass[
		book,
		head_space=20mm,
		foot_space=20mm,
		gutter=10mm,
		line_length=190mm
]{jlreq}

%----------
%LuaLaTeXで実行する!!
%----------
%各章節には以下を書く. 1-03.texのような名前にする
%----------
% \documentclass[
% 		book,
% 		head_space=20mm,
% 		foot_space=20mm,
% 		gutter=10mm,
% 		line_length=190mm
% ]{jlreq}
% 
%----------
%LuaLaTeXで実行する!!
%----------
%各章節には以下を書く. 1-03.texのような名前にする
%----------
% \documentclass[
% 		book,
% 		head_space=20mm,
% 		foot_space=20mm,
% 		gutter=10mm,
% 		line_length=190mm
% ]{jlreq}
% 
%----------
%LuaLaTeXで実行する!!
%----------
%各章節には以下を書く. 1-03.texのような名前にする
%----------
% \documentclass[
% 		book,
% 		head_space=20mm,
% 		foot_space=20mm,
% 		gutter=10mm,
% 		line_length=190mm
% ]{jlreq}
% \input {preamble.tex}
% \usepackage{docmute} %ファイル分割
% \begin{document}

% %\chapter{章のタイトル}
% \section{節のタイトル}
% no text

% \end{document}
%----------

%main.texには以下を書く
%----------
% \documentclass[
% 		book,
% 		head_space=20mm,
% 		foot_space=20mm,
% 		gutter=10mm,
% 		line_length=190mm,
%         openany
% ]{jlreq}
% \input {preamble.tex}
% \usepackage{docmute} %ファイル分割
% \begin{document}

% %---------- 1章1節
% \input 1-01.tex
% %---------- 1章2節
% \input 1-02.tex
% % ---------- 1章3節
% \input 1-03.tex
% % ---------- 1章4節
% \input 1-04.tex
% % ---------- 1章5節
% \input 1-05.tex
% % ---------- 1章6節
% \input 1-06.tex
% %---------- 1章7節
% \input 1-07.tex
% % ---------- 1章8節
% \input 1-08.tex
% % ---------- 1章9節
% \input 1-09.tex
% % ---------- 1章10節
% \input 1-10.tex
% % ---------- 1章11節
% \input 1-11.tex
% % ---------- 1章12節
% \input 1-12.tex
% % ---------- 参考文献
% \input reference.tex
% \end{document}
% ----------



\usepackage{bxtexlogo}
\usepackage{amsthm}
\usepackage{amsmath}
\usepackage{bbm} %小文字の黒板文字
\usepackage{physics}
\usepackage{amsfonts}
\usepackage{graphicx}
\usepackage{mathtools}
\usepackage{enumitem}
\usepackage[margin=20truemm]{geometry}
\usepackage{textcomp}
\usepackage{bm}
\usepackage{mathrsfs}
\usepackage{latexsym}
\usepackage{amssymb}
\usepackage{algorithmic}
\usepackage{algorithm}
\usepackage{tikz}
\usepackage{wrapfig}
\usetikzlibrary{arrows.meta}
\usetikzlibrary{math,matrix,backgrounds}
\usetikzlibrary{angles}
\usetikzlibrary{calc}


%----------
%日本語フォント
% \usepackage[deluxe]{otf} platex用 lualatexでは動かない

%----------
%欧文フォント
\usepackage[T1]{fontenc}

%----------
%文字色
\usepackage{color}

%----------
\setlength{\parindent}{2\zw} %インデントの設定

%----------
% %参照した数式にだけ番号を振る cleverrefと併用するとうまくいかない
% \mathtoolsset{showonlyrefs=true}
%----------

%----------
%集合の中線
\newcommand{\relmiddle}[1]{\mathrel{}\middle#1\mathrel{}}
% \middle| の代わりに \relmiddle| を付ける
\newcommand{\sgn}{\mathop{\mathrm{sgn}}} %置換sgn
\newcommand{\Int}{\mathop{\mathrm{Int}}} %位相空間の内部Int
\newcommand{\Ext}{\mathop{\mathrm{Ext}}} %位相空間の外部Ext
\newcommand{\Cl}{\mathop{\mathrm{Cl}}} %位相空間の閉包Cl
\newcommand{\supp}{\mathop{\mathrm{supp}}} %関数の台supp
\newcommand{\restrict}[2]{\left. #1 \right \vert_{#2}}%関数の制限 \restrict{f}{A} = f|_A
\newcommand{\Span}{\mathop{\mathrm{Span}}}
\newcommand{\Ker}{\mathop{\mathrm{Ker}}}
\newcommand{\Coker}{\mathop{\mathrm{Coker}}}
\newcommand{\coker}{\mathop{\mathrm{coker}}}
\newcommand{\Coim}{\mathop{\mathrm{Coim}}}
\newcommand{\coim}{\mathop{\mathrm{coim}}}
\newcommand{\id}{\mathop{\mathrm{id}}}
\newcommand{\Gal}{\mathop{\mathrm{Gal}}}
\renewcommand{\Im}{\mathop{\mathrm{Im}}}
\renewcommand{\Re}{\mathop{\mathrm{Re}}}


\newtheorem{definition}{定義}[section]

\usepackage{aliascnt}

% \newaliastheorem{(環境とカウンターの名前)}{(元となるカウンターの名前)}{(表示される文字列)}
\newcommand*{\newaliastheorem}[3]{%
  \newaliascnt{#1}{#2}%
  \newtheorem{#1}[#1]{#3}%
  \aliascntresetthe{#1}%
  \expandafter\newcommand\csname #1autorefname\endcsname{#3}%
}
\newaliastheorem{proposition}{definition}{命題} 
\newaliastheorem{theorem}{definition}{定理}
\newaliastheorem{lemma}{definition}{補題}
\newaliastheorem{corollary}{definition}{系}
\newaliastheorem{example}{definition}{例}
\newaliastheorem{practice}{definition}{演習問題}

\newtheorem*{longproof}{証明}
\newtheorem*{answer}{解答}
\newtheorem*{supplement}{補足}
\newtheorem*{remark}{注意}
%----------

%----------
%古い記法を注意するパッケージ
\RequirePackage[l2tabu, orthodox]{nag}
%----------


% 定理環境(tcolorbox)
\usepackage{tcolorbox} %箱
\tcbuselibrary{breakable,skins,theorems}
\tcolorboxenvironment{definition}{
	blanker,breakable,
	left=3mm,right=3mm,
	top=2mm,bottom=2mm,
	before skip=15pt,after skip=20pt,
	borderline ={0.5pt}{0pt}{black}
}
\newtcolorbox{emptydefinition}{
	blanker,breakable,
	left=3mm,right=3mm,
	top=2mm,bottom=2mm,
	before skip=15pt,after skip=20pt,
	borderline ={0.5pt}{0pt}{black}
}
%----------
\tcolorboxenvironment{proposition}{
	blanker,breakable,
	left=3mm,right=3mm,
	top=3mm,bottom=3mm,
	before skip=15pt,after skip=15pt,
	borderline={0.5pt}{0pt}{black}
}
\newtcolorbox{emptyproposition}{
	blanker,breakable,
	left=3mm,right=3mm,
	top=3mm,bottom=3mm,
	before skip=15pt,after skip=15pt,
	borderline={0.5pt}{0pt}{black}
}
%----------
\tcolorboxenvironment{theorem}{
	blanker,breakable,
	left=3mm,right=3mm,
	top=3mm,bottom=3mm,
    sharp corners,boxrule=0.6pt,
	before skip=15pt,after skip=15pt,
	borderline={0.5pt}{0pt}{black},
    borderline={0.5pt}{1.5pt}{black}
}
\newtcolorbox{emptytheorem}{
	blanker,breakable,
	left=3mm,right=3mm,
	top=3mm,bottom=3mm,
    sharp corners,boxrule=0.6pt,
	before skip=15pt,after skip=15pt,
	borderline={0.5pt}{0pt}{black},
    borderline={0.5pt}{1.5pt}{black}
}
%----------
\tcolorboxenvironment{lemma}{
	blanker,breakable,
	left=3mm,right=3mm,
	top=3mm,bottom=3mm,
	before skip=15pt,after skip=15pt,
	borderline={0.5pt}{0pt}{black}
}
%----------
\tcolorboxenvironment{corollary}{
	blanker,breakable,
	left=3mm,right=3mm,
	top=3mm,bottom=3mm,
	before skip=15pt,after skip=15pt,
	borderline={1.0pt}{0pt}{black,dotted}
}
\newtcolorbox{emptycorollary}{
	blanker,breakable,
	left=3mm,right=3mm,
	top=3mm,bottom=3mm,
	before skip=15pt,after skip=15pt,
	borderline={1.0pt}{0pt}{black,dotted}
}
%----------
\tcolorboxenvironment{example}{
	blanker,breakable,
	left=3mm,right=3mm,
	top=3mm,bottom=3mm,
	before skip=15pt,after skip=15pt,
	borderline={0.5pt}{0pt}{black}
}
%----------
\tcolorboxenvironment{practice}{
	blanker,breakable,
	left=3mm,right=3mm,
	top=3mm,bottom=3mm,
	before skip=15pt,after skip=15pt,
	borderline={0.5pt}{0pt}{black}
}
%----------
\tcolorboxenvironment{proof}{
	blanker,breakable,
	left=3mm,right=3mm,
	top=2mm,bottom=2mm,
	before skip=15pt,after skip=20pt,
	% borderline west={1.5pt}{0pt}{black,dotted}
	borderline vertical={1pt}{0pt}{black,dotted}
	% borderline vertical={0.8pt}{0pt}{black,dotted,arrows={Square[scale=0.5]-Square[scale=0.5]}}
	}
%----------
\tcolorboxenvironment{supplement}{
	blanker,breakable,
	left=3mm,right=3mm,
	top=2mm,bottom=2mm,
	before skip=15pt,after skip=20pt,
	% borderline west={1.5pt}{0pt}{black,dotted}
	% borderline vertical={0.5pt}{0pt}{black,arrows = {Circle[scale=0.7]-Circle[scale=0.7]}}
	borderline vertical={0.5pt}{0pt}{black}
	% borderline vertical={0.5pt}{0pt}{black},
	% borderline north={0.5pt}{0pt}{white,arrows={Circle[black,scale=0.7]-Circle[black,scale=0.7]}}
	}
%----------
\tcolorboxenvironment{remark}{
	blanker,breakable,
	left=3mm,right=3mm,
	top=1mm,bottom=1mm,
	before skip=15pt,after skip=20pt,
	% borderline west={1.5pt}{0pt}{black,dotted}
	% borderline vertical={0.5pt}{0pt}{black,arrows = {Circle[scale=0.7]-Circle[scale=0.7]}}
	borderline vertical={0.5pt}{0pt}{black}
	% borderline vertical={0.5pt}{0pt}{black},
	% borderline north={0.5pt}{0pt}{white,arrows={Circle[black,scale=0.7]-Circle[black,scale=0.7]}}
	}
    
%---------------------
 

%----------
%ハイパーリンク
% 「%」は以降の内容を「改行コードも含めて」無視するコマンド
\usepackage[%
%  dvipdfmx,% 欧文ではコメントアウトする
luatex,%
pdfencoding=auto,%
 setpagesize=false,%
 bookmarks=true,%
 bookmarksdepth=tocdepth,%
 bookmarksnumbered=true,%
 colorlinks=false,%
 pdftitle={},%
 pdfsubject={},%
 pdfauthor={},%
 pdfkeywords={}%
]{hyperref}
%------------


%参照 参照するときに自動で環境名を含んで参照する
\usepackage[nameinlink]{cleveref}
\let\normalref\ref
\renewcommand{\ref}{\cref}
\crefname{definition}{定義}{定義}
\crefname{proposition}{命題}{命題}
\crefname{theorem}{定理}{定理}
\crefname{lemma}{補題}{補題}
\crefname{corollary}{系}{系}
\crefname{example}{例}{例}
\crefname{practice}{演習問題}{演習問題}
\crefname{equation}{式}{式} 
\crefname{chapter}{第}{第}
\creflabelformat{chapter}{#2#1章#3}
\crefname{section}{第}{第}
\creflabelformat{section}{#2#1節#3}
\crefname{subsection}{第}{第}
\creflabelformat{subsection}{#2#1小節#3}
%----------

%---------------------
%章跨ぎの参照が不具合を起こすための代わり
% \mylabl でラベル付け
\newcommand{\mylabel}[1]{
\label{#1}
\hypertarget{#1}{}
}
% \myref で環境名付きリンクをつける
\newcommand{\myref}[1]{
\hyperlink{#1}{\cref*{#1}}
}
%-----------------

\usepackage{autonum} %参照した数式にだけ番号を振る
% \usepackage{docmute} %ファイル分割
% \begin{document}

% %\chapter{章のタイトル}
% \section{節のタイトル}
% no text

% \end{document}
%----------

%main.texには以下を書く
%----------
% \documentclass[
% 		book,
% 		head_space=20mm,
% 		foot_space=20mm,
% 		gutter=10mm,
% 		line_length=190mm,
%         openany
% ]{jlreq}
% 
%----------
%LuaLaTeXで実行する!!
%----------
%各章節には以下を書く. 1-03.texのような名前にする
%----------
% \documentclass[
% 		book,
% 		head_space=20mm,
% 		foot_space=20mm,
% 		gutter=10mm,
% 		line_length=190mm
% ]{jlreq}
% \input {preamble.tex}
% \usepackage{docmute} %ファイル分割
% \begin{document}

% %\chapter{章のタイトル}
% \section{節のタイトル}
% no text

% \end{document}
%----------

%main.texには以下を書く
%----------
% \documentclass[
% 		book,
% 		head_space=20mm,
% 		foot_space=20mm,
% 		gutter=10mm,
% 		line_length=190mm,
%         openany
% ]{jlreq}
% \input {preamble.tex}
% \usepackage{docmute} %ファイル分割
% \begin{document}

% %---------- 1章1節
% \input 1-01.tex
% %---------- 1章2節
% \input 1-02.tex
% % ---------- 1章3節
% \input 1-03.tex
% % ---------- 1章4節
% \input 1-04.tex
% % ---------- 1章5節
% \input 1-05.tex
% % ---------- 1章6節
% \input 1-06.tex
% %---------- 1章7節
% \input 1-07.tex
% % ---------- 1章8節
% \input 1-08.tex
% % ---------- 1章9節
% \input 1-09.tex
% % ---------- 1章10節
% \input 1-10.tex
% % ---------- 1章11節
% \input 1-11.tex
% % ---------- 1章12節
% \input 1-12.tex
% % ---------- 参考文献
% \input reference.tex
% \end{document}
% ----------



\usepackage{bxtexlogo}
\usepackage{amsthm}
\usepackage{amsmath}
\usepackage{bbm} %小文字の黒板文字
\usepackage{physics}
\usepackage{amsfonts}
\usepackage{graphicx}
\usepackage{mathtools}
\usepackage{enumitem}
\usepackage[margin=20truemm]{geometry}
\usepackage{textcomp}
\usepackage{bm}
\usepackage{mathrsfs}
\usepackage{latexsym}
\usepackage{amssymb}
\usepackage{algorithmic}
\usepackage{algorithm}
\usepackage{tikz}
\usepackage{wrapfig}
\usetikzlibrary{arrows.meta}
\usetikzlibrary{math,matrix,backgrounds}
\usetikzlibrary{angles}
\usetikzlibrary{calc}


%----------
%日本語フォント
% \usepackage[deluxe]{otf} platex用 lualatexでは動かない

%----------
%欧文フォント
\usepackage[T1]{fontenc}

%----------
%文字色
\usepackage{color}

%----------
\setlength{\parindent}{2\zw} %インデントの設定

%----------
% %参照した数式にだけ番号を振る cleverrefと併用するとうまくいかない
% \mathtoolsset{showonlyrefs=true}
%----------

%----------
%集合の中線
\newcommand{\relmiddle}[1]{\mathrel{}\middle#1\mathrel{}}
% \middle| の代わりに \relmiddle| を付ける
\newcommand{\sgn}{\mathop{\mathrm{sgn}}} %置換sgn
\newcommand{\Int}{\mathop{\mathrm{Int}}} %位相空間の内部Int
\newcommand{\Ext}{\mathop{\mathrm{Ext}}} %位相空間の外部Ext
\newcommand{\Cl}{\mathop{\mathrm{Cl}}} %位相空間の閉包Cl
\newcommand{\supp}{\mathop{\mathrm{supp}}} %関数の台supp
\newcommand{\restrict}[2]{\left. #1 \right \vert_{#2}}%関数の制限 \restrict{f}{A} = f|_A
\newcommand{\Span}{\mathop{\mathrm{Span}}}
\newcommand{\Ker}{\mathop{\mathrm{Ker}}}
\newcommand{\Coker}{\mathop{\mathrm{Coker}}}
\newcommand{\coker}{\mathop{\mathrm{coker}}}
\newcommand{\Coim}{\mathop{\mathrm{Coim}}}
\newcommand{\coim}{\mathop{\mathrm{coim}}}
\newcommand{\id}{\mathop{\mathrm{id}}}
\newcommand{\Gal}{\mathop{\mathrm{Gal}}}
\renewcommand{\Im}{\mathop{\mathrm{Im}}}
\renewcommand{\Re}{\mathop{\mathrm{Re}}}


\newtheorem{definition}{定義}[section]

\usepackage{aliascnt}

% \newaliastheorem{(環境とカウンターの名前)}{(元となるカウンターの名前)}{(表示される文字列)}
\newcommand*{\newaliastheorem}[3]{%
  \newaliascnt{#1}{#2}%
  \newtheorem{#1}[#1]{#3}%
  \aliascntresetthe{#1}%
  \expandafter\newcommand\csname #1autorefname\endcsname{#3}%
}
\newaliastheorem{proposition}{definition}{命題} 
\newaliastheorem{theorem}{definition}{定理}
\newaliastheorem{lemma}{definition}{補題}
\newaliastheorem{corollary}{definition}{系}
\newaliastheorem{example}{definition}{例}
\newaliastheorem{practice}{definition}{演習問題}

\newtheorem*{longproof}{証明}
\newtheorem*{answer}{解答}
\newtheorem*{supplement}{補足}
\newtheorem*{remark}{注意}
%----------

%----------
%古い記法を注意するパッケージ
\RequirePackage[l2tabu, orthodox]{nag}
%----------


% 定理環境(tcolorbox)
\usepackage{tcolorbox} %箱
\tcbuselibrary{breakable,skins,theorems}
\tcolorboxenvironment{definition}{
	blanker,breakable,
	left=3mm,right=3mm,
	top=2mm,bottom=2mm,
	before skip=15pt,after skip=20pt,
	borderline ={0.5pt}{0pt}{black}
}
\newtcolorbox{emptydefinition}{
	blanker,breakable,
	left=3mm,right=3mm,
	top=2mm,bottom=2mm,
	before skip=15pt,after skip=20pt,
	borderline ={0.5pt}{0pt}{black}
}
%----------
\tcolorboxenvironment{proposition}{
	blanker,breakable,
	left=3mm,right=3mm,
	top=3mm,bottom=3mm,
	before skip=15pt,after skip=15pt,
	borderline={0.5pt}{0pt}{black}
}
\newtcolorbox{emptyproposition}{
	blanker,breakable,
	left=3mm,right=3mm,
	top=3mm,bottom=3mm,
	before skip=15pt,after skip=15pt,
	borderline={0.5pt}{0pt}{black}
}
%----------
\tcolorboxenvironment{theorem}{
	blanker,breakable,
	left=3mm,right=3mm,
	top=3mm,bottom=3mm,
    sharp corners,boxrule=0.6pt,
	before skip=15pt,after skip=15pt,
	borderline={0.5pt}{0pt}{black},
    borderline={0.5pt}{1.5pt}{black}
}
\newtcolorbox{emptytheorem}{
	blanker,breakable,
	left=3mm,right=3mm,
	top=3mm,bottom=3mm,
    sharp corners,boxrule=0.6pt,
	before skip=15pt,after skip=15pt,
	borderline={0.5pt}{0pt}{black},
    borderline={0.5pt}{1.5pt}{black}
}
%----------
\tcolorboxenvironment{lemma}{
	blanker,breakable,
	left=3mm,right=3mm,
	top=3mm,bottom=3mm,
	before skip=15pt,after skip=15pt,
	borderline={0.5pt}{0pt}{black}
}
%----------
\tcolorboxenvironment{corollary}{
	blanker,breakable,
	left=3mm,right=3mm,
	top=3mm,bottom=3mm,
	before skip=15pt,after skip=15pt,
	borderline={1.0pt}{0pt}{black,dotted}
}
\newtcolorbox{emptycorollary}{
	blanker,breakable,
	left=3mm,right=3mm,
	top=3mm,bottom=3mm,
	before skip=15pt,after skip=15pt,
	borderline={1.0pt}{0pt}{black,dotted}
}
%----------
\tcolorboxenvironment{example}{
	blanker,breakable,
	left=3mm,right=3mm,
	top=3mm,bottom=3mm,
	before skip=15pt,after skip=15pt,
	borderline={0.5pt}{0pt}{black}
}
%----------
\tcolorboxenvironment{practice}{
	blanker,breakable,
	left=3mm,right=3mm,
	top=3mm,bottom=3mm,
	before skip=15pt,after skip=15pt,
	borderline={0.5pt}{0pt}{black}
}
%----------
\tcolorboxenvironment{proof}{
	blanker,breakable,
	left=3mm,right=3mm,
	top=2mm,bottom=2mm,
	before skip=15pt,after skip=20pt,
	% borderline west={1.5pt}{0pt}{black,dotted}
	borderline vertical={1pt}{0pt}{black,dotted}
	% borderline vertical={0.8pt}{0pt}{black,dotted,arrows={Square[scale=0.5]-Square[scale=0.5]}}
	}
%----------
\tcolorboxenvironment{supplement}{
	blanker,breakable,
	left=3mm,right=3mm,
	top=2mm,bottom=2mm,
	before skip=15pt,after skip=20pt,
	% borderline west={1.5pt}{0pt}{black,dotted}
	% borderline vertical={0.5pt}{0pt}{black,arrows = {Circle[scale=0.7]-Circle[scale=0.7]}}
	borderline vertical={0.5pt}{0pt}{black}
	% borderline vertical={0.5pt}{0pt}{black},
	% borderline north={0.5pt}{0pt}{white,arrows={Circle[black,scale=0.7]-Circle[black,scale=0.7]}}
	}
%----------
\tcolorboxenvironment{remark}{
	blanker,breakable,
	left=3mm,right=3mm,
	top=1mm,bottom=1mm,
	before skip=15pt,after skip=20pt,
	% borderline west={1.5pt}{0pt}{black,dotted}
	% borderline vertical={0.5pt}{0pt}{black,arrows = {Circle[scale=0.7]-Circle[scale=0.7]}}
	borderline vertical={0.5pt}{0pt}{black}
	% borderline vertical={0.5pt}{0pt}{black},
	% borderline north={0.5pt}{0pt}{white,arrows={Circle[black,scale=0.7]-Circle[black,scale=0.7]}}
	}
    
%---------------------
 

%----------
%ハイパーリンク
% 「%」は以降の内容を「改行コードも含めて」無視するコマンド
\usepackage[%
%  dvipdfmx,% 欧文ではコメントアウトする
luatex,%
pdfencoding=auto,%
 setpagesize=false,%
 bookmarks=true,%
 bookmarksdepth=tocdepth,%
 bookmarksnumbered=true,%
 colorlinks=false,%
 pdftitle={},%
 pdfsubject={},%
 pdfauthor={},%
 pdfkeywords={}%
]{hyperref}
%------------


%参照 参照するときに自動で環境名を含んで参照する
\usepackage[nameinlink]{cleveref}
\let\normalref\ref
\renewcommand{\ref}{\cref}
\crefname{definition}{定義}{定義}
\crefname{proposition}{命題}{命題}
\crefname{theorem}{定理}{定理}
\crefname{lemma}{補題}{補題}
\crefname{corollary}{系}{系}
\crefname{example}{例}{例}
\crefname{practice}{演習問題}{演習問題}
\crefname{equation}{式}{式} 
\crefname{chapter}{第}{第}
\creflabelformat{chapter}{#2#1章#3}
\crefname{section}{第}{第}
\creflabelformat{section}{#2#1節#3}
\crefname{subsection}{第}{第}
\creflabelformat{subsection}{#2#1小節#3}
%----------

%---------------------
%章跨ぎの参照が不具合を起こすための代わり
% \mylabl でラベル付け
\newcommand{\mylabel}[1]{
\label{#1}
\hypertarget{#1}{}
}
% \myref で環境名付きリンクをつける
\newcommand{\myref}[1]{
\hyperlink{#1}{\cref*{#1}}
}
%-----------------

\usepackage{autonum} %参照した数式にだけ番号を振る
% \usepackage{docmute} %ファイル分割
% \begin{document}

% %---------- 1章1節
% \input 1-01.tex
% %---------- 1章2節
% \input 1-02.tex
% % ---------- 1章3節
% \input 1-03.tex
% % ---------- 1章4節
% \input 1-04.tex
% % ---------- 1章5節
% \input 1-05.tex
% % ---------- 1章6節
% \input 1-06.tex
% %---------- 1章7節
% \input 1-07.tex
% % ---------- 1章8節
% \input 1-08.tex
% % ---------- 1章9節
% \input 1-09.tex
% % ---------- 1章10節
% \input 1-10.tex
% % ---------- 1章11節
% \input 1-11.tex
% % ---------- 1章12節
% \input 1-12.tex
% % ---------- 参考文献
% \input reference.tex
% \end{document}
% ----------



\usepackage{bxtexlogo}
\usepackage{amsthm}
\usepackage{amsmath}
\usepackage{bbm} %小文字の黒板文字
\usepackage{physics}
\usepackage{amsfonts}
\usepackage{graphicx}
\usepackage{mathtools}
\usepackage{enumitem}
\usepackage[margin=20truemm]{geometry}
\usepackage{textcomp}
\usepackage{bm}
\usepackage{mathrsfs}
\usepackage{latexsym}
\usepackage{amssymb}
\usepackage{algorithmic}
\usepackage{algorithm}
\usepackage{tikz}
\usepackage{wrapfig}
\usetikzlibrary{arrows.meta}
\usetikzlibrary{math,matrix,backgrounds}
\usetikzlibrary{angles}
\usetikzlibrary{calc}


%----------
%日本語フォント
% \usepackage[deluxe]{otf} platex用 lualatexでは動かない

%----------
%欧文フォント
\usepackage[T1]{fontenc}

%----------
%文字色
\usepackage{color}

%----------
\setlength{\parindent}{2\zw} %インデントの設定

%----------
% %参照した数式にだけ番号を振る cleverrefと併用するとうまくいかない
% \mathtoolsset{showonlyrefs=true}
%----------

%----------
%集合の中線
\newcommand{\relmiddle}[1]{\mathrel{}\middle#1\mathrel{}}
% \middle| の代わりに \relmiddle| を付ける
\newcommand{\sgn}{\mathop{\mathrm{sgn}}} %置換sgn
\newcommand{\Int}{\mathop{\mathrm{Int}}} %位相空間の内部Int
\newcommand{\Ext}{\mathop{\mathrm{Ext}}} %位相空間の外部Ext
\newcommand{\Cl}{\mathop{\mathrm{Cl}}} %位相空間の閉包Cl
\newcommand{\supp}{\mathop{\mathrm{supp}}} %関数の台supp
\newcommand{\restrict}[2]{\left. #1 \right \vert_{#2}}%関数の制限 \restrict{f}{A} = f|_A
\newcommand{\Span}{\mathop{\mathrm{Span}}}
\newcommand{\Ker}{\mathop{\mathrm{Ker}}}
\newcommand{\Coker}{\mathop{\mathrm{Coker}}}
\newcommand{\coker}{\mathop{\mathrm{coker}}}
\newcommand{\Coim}{\mathop{\mathrm{Coim}}}
\newcommand{\coim}{\mathop{\mathrm{coim}}}
\newcommand{\id}{\mathop{\mathrm{id}}}
\newcommand{\Gal}{\mathop{\mathrm{Gal}}}
\renewcommand{\Im}{\mathop{\mathrm{Im}}}
\renewcommand{\Re}{\mathop{\mathrm{Re}}}


\newtheorem{definition}{定義}[section]

\usepackage{aliascnt}

% \newaliastheorem{(環境とカウンターの名前)}{(元となるカウンターの名前)}{(表示される文字列)}
\newcommand*{\newaliastheorem}[3]{%
  \newaliascnt{#1}{#2}%
  \newtheorem{#1}[#1]{#3}%
  \aliascntresetthe{#1}%
  \expandafter\newcommand\csname #1autorefname\endcsname{#3}%
}
\newaliastheorem{proposition}{definition}{命題} 
\newaliastheorem{theorem}{definition}{定理}
\newaliastheorem{lemma}{definition}{補題}
\newaliastheorem{corollary}{definition}{系}
\newaliastheorem{example}{definition}{例}
\newaliastheorem{practice}{definition}{演習問題}

\newtheorem*{longproof}{証明}
\newtheorem*{answer}{解答}
\newtheorem*{supplement}{補足}
\newtheorem*{remark}{注意}
%----------

%----------
%古い記法を注意するパッケージ
\RequirePackage[l2tabu, orthodox]{nag}
%----------


% 定理環境(tcolorbox)
\usepackage{tcolorbox} %箱
\tcbuselibrary{breakable,skins,theorems}
\tcolorboxenvironment{definition}{
	blanker,breakable,
	left=3mm,right=3mm,
	top=2mm,bottom=2mm,
	before skip=15pt,after skip=20pt,
	borderline ={0.5pt}{0pt}{black}
}
\newtcolorbox{emptydefinition}{
	blanker,breakable,
	left=3mm,right=3mm,
	top=2mm,bottom=2mm,
	before skip=15pt,after skip=20pt,
	borderline ={0.5pt}{0pt}{black}
}
%----------
\tcolorboxenvironment{proposition}{
	blanker,breakable,
	left=3mm,right=3mm,
	top=3mm,bottom=3mm,
	before skip=15pt,after skip=15pt,
	borderline={0.5pt}{0pt}{black}
}
\newtcolorbox{emptyproposition}{
	blanker,breakable,
	left=3mm,right=3mm,
	top=3mm,bottom=3mm,
	before skip=15pt,after skip=15pt,
	borderline={0.5pt}{0pt}{black}
}
%----------
\tcolorboxenvironment{theorem}{
	blanker,breakable,
	left=3mm,right=3mm,
	top=3mm,bottom=3mm,
    sharp corners,boxrule=0.6pt,
	before skip=15pt,after skip=15pt,
	borderline={0.5pt}{0pt}{black},
    borderline={0.5pt}{1.5pt}{black}
}
\newtcolorbox{emptytheorem}{
	blanker,breakable,
	left=3mm,right=3mm,
	top=3mm,bottom=3mm,
    sharp corners,boxrule=0.6pt,
	before skip=15pt,after skip=15pt,
	borderline={0.5pt}{0pt}{black},
    borderline={0.5pt}{1.5pt}{black}
}
%----------
\tcolorboxenvironment{lemma}{
	blanker,breakable,
	left=3mm,right=3mm,
	top=3mm,bottom=3mm,
	before skip=15pt,after skip=15pt,
	borderline={0.5pt}{0pt}{black}
}
%----------
\tcolorboxenvironment{corollary}{
	blanker,breakable,
	left=3mm,right=3mm,
	top=3mm,bottom=3mm,
	before skip=15pt,after skip=15pt,
	borderline={1.0pt}{0pt}{black,dotted}
}
\newtcolorbox{emptycorollary}{
	blanker,breakable,
	left=3mm,right=3mm,
	top=3mm,bottom=3mm,
	before skip=15pt,after skip=15pt,
	borderline={1.0pt}{0pt}{black,dotted}
}
%----------
\tcolorboxenvironment{example}{
	blanker,breakable,
	left=3mm,right=3mm,
	top=3mm,bottom=3mm,
	before skip=15pt,after skip=15pt,
	borderline={0.5pt}{0pt}{black}
}
%----------
\tcolorboxenvironment{practice}{
	blanker,breakable,
	left=3mm,right=3mm,
	top=3mm,bottom=3mm,
	before skip=15pt,after skip=15pt,
	borderline={0.5pt}{0pt}{black}
}
%----------
\tcolorboxenvironment{proof}{
	blanker,breakable,
	left=3mm,right=3mm,
	top=2mm,bottom=2mm,
	before skip=15pt,after skip=20pt,
	% borderline west={1.5pt}{0pt}{black,dotted}
	borderline vertical={1pt}{0pt}{black,dotted}
	% borderline vertical={0.8pt}{0pt}{black,dotted,arrows={Square[scale=0.5]-Square[scale=0.5]}}
	}
%----------
\tcolorboxenvironment{supplement}{
	blanker,breakable,
	left=3mm,right=3mm,
	top=2mm,bottom=2mm,
	before skip=15pt,after skip=20pt,
	% borderline west={1.5pt}{0pt}{black,dotted}
	% borderline vertical={0.5pt}{0pt}{black,arrows = {Circle[scale=0.7]-Circle[scale=0.7]}}
	borderline vertical={0.5pt}{0pt}{black}
	% borderline vertical={0.5pt}{0pt}{black},
	% borderline north={0.5pt}{0pt}{white,arrows={Circle[black,scale=0.7]-Circle[black,scale=0.7]}}
	}
%----------
\tcolorboxenvironment{remark}{
	blanker,breakable,
	left=3mm,right=3mm,
	top=1mm,bottom=1mm,
	before skip=15pt,after skip=20pt,
	% borderline west={1.5pt}{0pt}{black,dotted}
	% borderline vertical={0.5pt}{0pt}{black,arrows = {Circle[scale=0.7]-Circle[scale=0.7]}}
	borderline vertical={0.5pt}{0pt}{black}
	% borderline vertical={0.5pt}{0pt}{black},
	% borderline north={0.5pt}{0pt}{white,arrows={Circle[black,scale=0.7]-Circle[black,scale=0.7]}}
	}
    
%---------------------
 

%----------
%ハイパーリンク
% 「%」は以降の内容を「改行コードも含めて」無視するコマンド
\usepackage[%
%  dvipdfmx,% 欧文ではコメントアウトする
luatex,%
pdfencoding=auto,%
 setpagesize=false,%
 bookmarks=true,%
 bookmarksdepth=tocdepth,%
 bookmarksnumbered=true,%
 colorlinks=false,%
 pdftitle={},%
 pdfsubject={},%
 pdfauthor={},%
 pdfkeywords={}%
]{hyperref}
%------------


%参照 参照するときに自動で環境名を含んで参照する
\usepackage[nameinlink]{cleveref}
\let\normalref\ref
\renewcommand{\ref}{\cref}
\crefname{definition}{定義}{定義}
\crefname{proposition}{命題}{命題}
\crefname{theorem}{定理}{定理}
\crefname{lemma}{補題}{補題}
\crefname{corollary}{系}{系}
\crefname{example}{例}{例}
\crefname{practice}{演習問題}{演習問題}
\crefname{equation}{式}{式} 
\crefname{chapter}{第}{第}
\creflabelformat{chapter}{#2#1章#3}
\crefname{section}{第}{第}
\creflabelformat{section}{#2#1節#3}
\crefname{subsection}{第}{第}
\creflabelformat{subsection}{#2#1小節#3}
%----------

%---------------------
%章跨ぎの参照が不具合を起こすための代わり
% \mylabl でラベル付け
\newcommand{\mylabel}[1]{
\label{#1}
\hypertarget{#1}{}
}
% \myref で環境名付きリンクをつける
\newcommand{\myref}[1]{
\hyperlink{#1}{\cref*{#1}}
}
%-----------------

\usepackage{autonum} %参照した数式にだけ番号を振る
% \usepackage{docmute} %ファイル分割
% \begin{document}

% %\chapter{章のタイトル}
% \section{節のタイトル}
% no text

% \end{document}
%----------

%main.texには以下を書く
%----------
% \documentclass[
% 		book,
% 		head_space=20mm,
% 		foot_space=20mm,
% 		gutter=10mm,
% 		line_length=190mm,
%         openany
% ]{jlreq}
% 
%----------
%LuaLaTeXで実行する!!
%----------
%各章節には以下を書く. 1-03.texのような名前にする
%----------
% \documentclass[
% 		book,
% 		head_space=20mm,
% 		foot_space=20mm,
% 		gutter=10mm,
% 		line_length=190mm
% ]{jlreq}
% 
%----------
%LuaLaTeXで実行する!!
%----------
%各章節には以下を書く. 1-03.texのような名前にする
%----------
% \documentclass[
% 		book,
% 		head_space=20mm,
% 		foot_space=20mm,
% 		gutter=10mm,
% 		line_length=190mm
% ]{jlreq}
% \input {preamble.tex}
% \usepackage{docmute} %ファイル分割
% \begin{document}

% %\chapter{章のタイトル}
% \section{節のタイトル}
% no text

% \end{document}
%----------

%main.texには以下を書く
%----------
% \documentclass[
% 		book,
% 		head_space=20mm,
% 		foot_space=20mm,
% 		gutter=10mm,
% 		line_length=190mm,
%         openany
% ]{jlreq}
% \input {preamble.tex}
% \usepackage{docmute} %ファイル分割
% \begin{document}

% %---------- 1章1節
% \input 1-01.tex
% %---------- 1章2節
% \input 1-02.tex
% % ---------- 1章3節
% \input 1-03.tex
% % ---------- 1章4節
% \input 1-04.tex
% % ---------- 1章5節
% \input 1-05.tex
% % ---------- 1章6節
% \input 1-06.tex
% %---------- 1章7節
% \input 1-07.tex
% % ---------- 1章8節
% \input 1-08.tex
% % ---------- 1章9節
% \input 1-09.tex
% % ---------- 1章10節
% \input 1-10.tex
% % ---------- 1章11節
% \input 1-11.tex
% % ---------- 1章12節
% \input 1-12.tex
% % ---------- 参考文献
% \input reference.tex
% \end{document}
% ----------



\usepackage{bxtexlogo}
\usepackage{amsthm}
\usepackage{amsmath}
\usepackage{bbm} %小文字の黒板文字
\usepackage{physics}
\usepackage{amsfonts}
\usepackage{graphicx}
\usepackage{mathtools}
\usepackage{enumitem}
\usepackage[margin=20truemm]{geometry}
\usepackage{textcomp}
\usepackage{bm}
\usepackage{mathrsfs}
\usepackage{latexsym}
\usepackage{amssymb}
\usepackage{algorithmic}
\usepackage{algorithm}
\usepackage{tikz}
\usepackage{wrapfig}
\usetikzlibrary{arrows.meta}
\usetikzlibrary{math,matrix,backgrounds}
\usetikzlibrary{angles}
\usetikzlibrary{calc}


%----------
%日本語フォント
% \usepackage[deluxe]{otf} platex用 lualatexでは動かない

%----------
%欧文フォント
\usepackage[T1]{fontenc}

%----------
%文字色
\usepackage{color}

%----------
\setlength{\parindent}{2\zw} %インデントの設定

%----------
% %参照した数式にだけ番号を振る cleverrefと併用するとうまくいかない
% \mathtoolsset{showonlyrefs=true}
%----------

%----------
%集合の中線
\newcommand{\relmiddle}[1]{\mathrel{}\middle#1\mathrel{}}
% \middle| の代わりに \relmiddle| を付ける
\newcommand{\sgn}{\mathop{\mathrm{sgn}}} %置換sgn
\newcommand{\Int}{\mathop{\mathrm{Int}}} %位相空間の内部Int
\newcommand{\Ext}{\mathop{\mathrm{Ext}}} %位相空間の外部Ext
\newcommand{\Cl}{\mathop{\mathrm{Cl}}} %位相空間の閉包Cl
\newcommand{\supp}{\mathop{\mathrm{supp}}} %関数の台supp
\newcommand{\restrict}[2]{\left. #1 \right \vert_{#2}}%関数の制限 \restrict{f}{A} = f|_A
\newcommand{\Span}{\mathop{\mathrm{Span}}}
\newcommand{\Ker}{\mathop{\mathrm{Ker}}}
\newcommand{\Coker}{\mathop{\mathrm{Coker}}}
\newcommand{\coker}{\mathop{\mathrm{coker}}}
\newcommand{\Coim}{\mathop{\mathrm{Coim}}}
\newcommand{\coim}{\mathop{\mathrm{coim}}}
\newcommand{\id}{\mathop{\mathrm{id}}}
\newcommand{\Gal}{\mathop{\mathrm{Gal}}}
\renewcommand{\Im}{\mathop{\mathrm{Im}}}
\renewcommand{\Re}{\mathop{\mathrm{Re}}}


\newtheorem{definition}{定義}[section]

\usepackage{aliascnt}

% \newaliastheorem{(環境とカウンターの名前)}{(元となるカウンターの名前)}{(表示される文字列)}
\newcommand*{\newaliastheorem}[3]{%
  \newaliascnt{#1}{#2}%
  \newtheorem{#1}[#1]{#3}%
  \aliascntresetthe{#1}%
  \expandafter\newcommand\csname #1autorefname\endcsname{#3}%
}
\newaliastheorem{proposition}{definition}{命題} 
\newaliastheorem{theorem}{definition}{定理}
\newaliastheorem{lemma}{definition}{補題}
\newaliastheorem{corollary}{definition}{系}
\newaliastheorem{example}{definition}{例}
\newaliastheorem{practice}{definition}{演習問題}

\newtheorem*{longproof}{証明}
\newtheorem*{answer}{解答}
\newtheorem*{supplement}{補足}
\newtheorem*{remark}{注意}
%----------

%----------
%古い記法を注意するパッケージ
\RequirePackage[l2tabu, orthodox]{nag}
%----------


% 定理環境(tcolorbox)
\usepackage{tcolorbox} %箱
\tcbuselibrary{breakable,skins,theorems}
\tcolorboxenvironment{definition}{
	blanker,breakable,
	left=3mm,right=3mm,
	top=2mm,bottom=2mm,
	before skip=15pt,after skip=20pt,
	borderline ={0.5pt}{0pt}{black}
}
\newtcolorbox{emptydefinition}{
	blanker,breakable,
	left=3mm,right=3mm,
	top=2mm,bottom=2mm,
	before skip=15pt,after skip=20pt,
	borderline ={0.5pt}{0pt}{black}
}
%----------
\tcolorboxenvironment{proposition}{
	blanker,breakable,
	left=3mm,right=3mm,
	top=3mm,bottom=3mm,
	before skip=15pt,after skip=15pt,
	borderline={0.5pt}{0pt}{black}
}
\newtcolorbox{emptyproposition}{
	blanker,breakable,
	left=3mm,right=3mm,
	top=3mm,bottom=3mm,
	before skip=15pt,after skip=15pt,
	borderline={0.5pt}{0pt}{black}
}
%----------
\tcolorboxenvironment{theorem}{
	blanker,breakable,
	left=3mm,right=3mm,
	top=3mm,bottom=3mm,
    sharp corners,boxrule=0.6pt,
	before skip=15pt,after skip=15pt,
	borderline={0.5pt}{0pt}{black},
    borderline={0.5pt}{1.5pt}{black}
}
\newtcolorbox{emptytheorem}{
	blanker,breakable,
	left=3mm,right=3mm,
	top=3mm,bottom=3mm,
    sharp corners,boxrule=0.6pt,
	before skip=15pt,after skip=15pt,
	borderline={0.5pt}{0pt}{black},
    borderline={0.5pt}{1.5pt}{black}
}
%----------
\tcolorboxenvironment{lemma}{
	blanker,breakable,
	left=3mm,right=3mm,
	top=3mm,bottom=3mm,
	before skip=15pt,after skip=15pt,
	borderline={0.5pt}{0pt}{black}
}
%----------
\tcolorboxenvironment{corollary}{
	blanker,breakable,
	left=3mm,right=3mm,
	top=3mm,bottom=3mm,
	before skip=15pt,after skip=15pt,
	borderline={1.0pt}{0pt}{black,dotted}
}
\newtcolorbox{emptycorollary}{
	blanker,breakable,
	left=3mm,right=3mm,
	top=3mm,bottom=3mm,
	before skip=15pt,after skip=15pt,
	borderline={1.0pt}{0pt}{black,dotted}
}
%----------
\tcolorboxenvironment{example}{
	blanker,breakable,
	left=3mm,right=3mm,
	top=3mm,bottom=3mm,
	before skip=15pt,after skip=15pt,
	borderline={0.5pt}{0pt}{black}
}
%----------
\tcolorboxenvironment{practice}{
	blanker,breakable,
	left=3mm,right=3mm,
	top=3mm,bottom=3mm,
	before skip=15pt,after skip=15pt,
	borderline={0.5pt}{0pt}{black}
}
%----------
\tcolorboxenvironment{proof}{
	blanker,breakable,
	left=3mm,right=3mm,
	top=2mm,bottom=2mm,
	before skip=15pt,after skip=20pt,
	% borderline west={1.5pt}{0pt}{black,dotted}
	borderline vertical={1pt}{0pt}{black,dotted}
	% borderline vertical={0.8pt}{0pt}{black,dotted,arrows={Square[scale=0.5]-Square[scale=0.5]}}
	}
%----------
\tcolorboxenvironment{supplement}{
	blanker,breakable,
	left=3mm,right=3mm,
	top=2mm,bottom=2mm,
	before skip=15pt,after skip=20pt,
	% borderline west={1.5pt}{0pt}{black,dotted}
	% borderline vertical={0.5pt}{0pt}{black,arrows = {Circle[scale=0.7]-Circle[scale=0.7]}}
	borderline vertical={0.5pt}{0pt}{black}
	% borderline vertical={0.5pt}{0pt}{black},
	% borderline north={0.5pt}{0pt}{white,arrows={Circle[black,scale=0.7]-Circle[black,scale=0.7]}}
	}
%----------
\tcolorboxenvironment{remark}{
	blanker,breakable,
	left=3mm,right=3mm,
	top=1mm,bottom=1mm,
	before skip=15pt,after skip=20pt,
	% borderline west={1.5pt}{0pt}{black,dotted}
	% borderline vertical={0.5pt}{0pt}{black,arrows = {Circle[scale=0.7]-Circle[scale=0.7]}}
	borderline vertical={0.5pt}{0pt}{black}
	% borderline vertical={0.5pt}{0pt}{black},
	% borderline north={0.5pt}{0pt}{white,arrows={Circle[black,scale=0.7]-Circle[black,scale=0.7]}}
	}
    
%---------------------
 

%----------
%ハイパーリンク
% 「%」は以降の内容を「改行コードも含めて」無視するコマンド
\usepackage[%
%  dvipdfmx,% 欧文ではコメントアウトする
luatex,%
pdfencoding=auto,%
 setpagesize=false,%
 bookmarks=true,%
 bookmarksdepth=tocdepth,%
 bookmarksnumbered=true,%
 colorlinks=false,%
 pdftitle={},%
 pdfsubject={},%
 pdfauthor={},%
 pdfkeywords={}%
]{hyperref}
%------------


%参照 参照するときに自動で環境名を含んで参照する
\usepackage[nameinlink]{cleveref}
\let\normalref\ref
\renewcommand{\ref}{\cref}
\crefname{definition}{定義}{定義}
\crefname{proposition}{命題}{命題}
\crefname{theorem}{定理}{定理}
\crefname{lemma}{補題}{補題}
\crefname{corollary}{系}{系}
\crefname{example}{例}{例}
\crefname{practice}{演習問題}{演習問題}
\crefname{equation}{式}{式} 
\crefname{chapter}{第}{第}
\creflabelformat{chapter}{#2#1章#3}
\crefname{section}{第}{第}
\creflabelformat{section}{#2#1節#3}
\crefname{subsection}{第}{第}
\creflabelformat{subsection}{#2#1小節#3}
%----------

%---------------------
%章跨ぎの参照が不具合を起こすための代わり
% \mylabl でラベル付け
\newcommand{\mylabel}[1]{
\label{#1}
\hypertarget{#1}{}
}
% \myref で環境名付きリンクをつける
\newcommand{\myref}[1]{
\hyperlink{#1}{\cref*{#1}}
}
%-----------------

\usepackage{autonum} %参照した数式にだけ番号を振る
% \usepackage{docmute} %ファイル分割
% \begin{document}

% %\chapter{章のタイトル}
% \section{節のタイトル}
% no text

% \end{document}
%----------

%main.texには以下を書く
%----------
% \documentclass[
% 		book,
% 		head_space=20mm,
% 		foot_space=20mm,
% 		gutter=10mm,
% 		line_length=190mm,
%         openany
% ]{jlreq}
% 
%----------
%LuaLaTeXで実行する!!
%----------
%各章節には以下を書く. 1-03.texのような名前にする
%----------
% \documentclass[
% 		book,
% 		head_space=20mm,
% 		foot_space=20mm,
% 		gutter=10mm,
% 		line_length=190mm
% ]{jlreq}
% \input {preamble.tex}
% \usepackage{docmute} %ファイル分割
% \begin{document}

% %\chapter{章のタイトル}
% \section{節のタイトル}
% no text

% \end{document}
%----------

%main.texには以下を書く
%----------
% \documentclass[
% 		book,
% 		head_space=20mm,
% 		foot_space=20mm,
% 		gutter=10mm,
% 		line_length=190mm,
%         openany
% ]{jlreq}
% \input {preamble.tex}
% \usepackage{docmute} %ファイル分割
% \begin{document}

% %---------- 1章1節
% \input 1-01.tex
% %---------- 1章2節
% \input 1-02.tex
% % ---------- 1章3節
% \input 1-03.tex
% % ---------- 1章4節
% \input 1-04.tex
% % ---------- 1章5節
% \input 1-05.tex
% % ---------- 1章6節
% \input 1-06.tex
% %---------- 1章7節
% \input 1-07.tex
% % ---------- 1章8節
% \input 1-08.tex
% % ---------- 1章9節
% \input 1-09.tex
% % ---------- 1章10節
% \input 1-10.tex
% % ---------- 1章11節
% \input 1-11.tex
% % ---------- 1章12節
% \input 1-12.tex
% % ---------- 参考文献
% \input reference.tex
% \end{document}
% ----------



\usepackage{bxtexlogo}
\usepackage{amsthm}
\usepackage{amsmath}
\usepackage{bbm} %小文字の黒板文字
\usepackage{physics}
\usepackage{amsfonts}
\usepackage{graphicx}
\usepackage{mathtools}
\usepackage{enumitem}
\usepackage[margin=20truemm]{geometry}
\usepackage{textcomp}
\usepackage{bm}
\usepackage{mathrsfs}
\usepackage{latexsym}
\usepackage{amssymb}
\usepackage{algorithmic}
\usepackage{algorithm}
\usepackage{tikz}
\usepackage{wrapfig}
\usetikzlibrary{arrows.meta}
\usetikzlibrary{math,matrix,backgrounds}
\usetikzlibrary{angles}
\usetikzlibrary{calc}


%----------
%日本語フォント
% \usepackage[deluxe]{otf} platex用 lualatexでは動かない

%----------
%欧文フォント
\usepackage[T1]{fontenc}

%----------
%文字色
\usepackage{color}

%----------
\setlength{\parindent}{2\zw} %インデントの設定

%----------
% %参照した数式にだけ番号を振る cleverrefと併用するとうまくいかない
% \mathtoolsset{showonlyrefs=true}
%----------

%----------
%集合の中線
\newcommand{\relmiddle}[1]{\mathrel{}\middle#1\mathrel{}}
% \middle| の代わりに \relmiddle| を付ける
\newcommand{\sgn}{\mathop{\mathrm{sgn}}} %置換sgn
\newcommand{\Int}{\mathop{\mathrm{Int}}} %位相空間の内部Int
\newcommand{\Ext}{\mathop{\mathrm{Ext}}} %位相空間の外部Ext
\newcommand{\Cl}{\mathop{\mathrm{Cl}}} %位相空間の閉包Cl
\newcommand{\supp}{\mathop{\mathrm{supp}}} %関数の台supp
\newcommand{\restrict}[2]{\left. #1 \right \vert_{#2}}%関数の制限 \restrict{f}{A} = f|_A
\newcommand{\Span}{\mathop{\mathrm{Span}}}
\newcommand{\Ker}{\mathop{\mathrm{Ker}}}
\newcommand{\Coker}{\mathop{\mathrm{Coker}}}
\newcommand{\coker}{\mathop{\mathrm{coker}}}
\newcommand{\Coim}{\mathop{\mathrm{Coim}}}
\newcommand{\coim}{\mathop{\mathrm{coim}}}
\newcommand{\id}{\mathop{\mathrm{id}}}
\newcommand{\Gal}{\mathop{\mathrm{Gal}}}
\renewcommand{\Im}{\mathop{\mathrm{Im}}}
\renewcommand{\Re}{\mathop{\mathrm{Re}}}


\newtheorem{definition}{定義}[section]

\usepackage{aliascnt}

% \newaliastheorem{(環境とカウンターの名前)}{(元となるカウンターの名前)}{(表示される文字列)}
\newcommand*{\newaliastheorem}[3]{%
  \newaliascnt{#1}{#2}%
  \newtheorem{#1}[#1]{#3}%
  \aliascntresetthe{#1}%
  \expandafter\newcommand\csname #1autorefname\endcsname{#3}%
}
\newaliastheorem{proposition}{definition}{命題} 
\newaliastheorem{theorem}{definition}{定理}
\newaliastheorem{lemma}{definition}{補題}
\newaliastheorem{corollary}{definition}{系}
\newaliastheorem{example}{definition}{例}
\newaliastheorem{practice}{definition}{演習問題}

\newtheorem*{longproof}{証明}
\newtheorem*{answer}{解答}
\newtheorem*{supplement}{補足}
\newtheorem*{remark}{注意}
%----------

%----------
%古い記法を注意するパッケージ
\RequirePackage[l2tabu, orthodox]{nag}
%----------


% 定理環境(tcolorbox)
\usepackage{tcolorbox} %箱
\tcbuselibrary{breakable,skins,theorems}
\tcolorboxenvironment{definition}{
	blanker,breakable,
	left=3mm,right=3mm,
	top=2mm,bottom=2mm,
	before skip=15pt,after skip=20pt,
	borderline ={0.5pt}{0pt}{black}
}
\newtcolorbox{emptydefinition}{
	blanker,breakable,
	left=3mm,right=3mm,
	top=2mm,bottom=2mm,
	before skip=15pt,after skip=20pt,
	borderline ={0.5pt}{0pt}{black}
}
%----------
\tcolorboxenvironment{proposition}{
	blanker,breakable,
	left=3mm,right=3mm,
	top=3mm,bottom=3mm,
	before skip=15pt,after skip=15pt,
	borderline={0.5pt}{0pt}{black}
}
\newtcolorbox{emptyproposition}{
	blanker,breakable,
	left=3mm,right=3mm,
	top=3mm,bottom=3mm,
	before skip=15pt,after skip=15pt,
	borderline={0.5pt}{0pt}{black}
}
%----------
\tcolorboxenvironment{theorem}{
	blanker,breakable,
	left=3mm,right=3mm,
	top=3mm,bottom=3mm,
    sharp corners,boxrule=0.6pt,
	before skip=15pt,after skip=15pt,
	borderline={0.5pt}{0pt}{black},
    borderline={0.5pt}{1.5pt}{black}
}
\newtcolorbox{emptytheorem}{
	blanker,breakable,
	left=3mm,right=3mm,
	top=3mm,bottom=3mm,
    sharp corners,boxrule=0.6pt,
	before skip=15pt,after skip=15pt,
	borderline={0.5pt}{0pt}{black},
    borderline={0.5pt}{1.5pt}{black}
}
%----------
\tcolorboxenvironment{lemma}{
	blanker,breakable,
	left=3mm,right=3mm,
	top=3mm,bottom=3mm,
	before skip=15pt,after skip=15pt,
	borderline={0.5pt}{0pt}{black}
}
%----------
\tcolorboxenvironment{corollary}{
	blanker,breakable,
	left=3mm,right=3mm,
	top=3mm,bottom=3mm,
	before skip=15pt,after skip=15pt,
	borderline={1.0pt}{0pt}{black,dotted}
}
\newtcolorbox{emptycorollary}{
	blanker,breakable,
	left=3mm,right=3mm,
	top=3mm,bottom=3mm,
	before skip=15pt,after skip=15pt,
	borderline={1.0pt}{0pt}{black,dotted}
}
%----------
\tcolorboxenvironment{example}{
	blanker,breakable,
	left=3mm,right=3mm,
	top=3mm,bottom=3mm,
	before skip=15pt,after skip=15pt,
	borderline={0.5pt}{0pt}{black}
}
%----------
\tcolorboxenvironment{practice}{
	blanker,breakable,
	left=3mm,right=3mm,
	top=3mm,bottom=3mm,
	before skip=15pt,after skip=15pt,
	borderline={0.5pt}{0pt}{black}
}
%----------
\tcolorboxenvironment{proof}{
	blanker,breakable,
	left=3mm,right=3mm,
	top=2mm,bottom=2mm,
	before skip=15pt,after skip=20pt,
	% borderline west={1.5pt}{0pt}{black,dotted}
	borderline vertical={1pt}{0pt}{black,dotted}
	% borderline vertical={0.8pt}{0pt}{black,dotted,arrows={Square[scale=0.5]-Square[scale=0.5]}}
	}
%----------
\tcolorboxenvironment{supplement}{
	blanker,breakable,
	left=3mm,right=3mm,
	top=2mm,bottom=2mm,
	before skip=15pt,after skip=20pt,
	% borderline west={1.5pt}{0pt}{black,dotted}
	% borderline vertical={0.5pt}{0pt}{black,arrows = {Circle[scale=0.7]-Circle[scale=0.7]}}
	borderline vertical={0.5pt}{0pt}{black}
	% borderline vertical={0.5pt}{0pt}{black},
	% borderline north={0.5pt}{0pt}{white,arrows={Circle[black,scale=0.7]-Circle[black,scale=0.7]}}
	}
%----------
\tcolorboxenvironment{remark}{
	blanker,breakable,
	left=3mm,right=3mm,
	top=1mm,bottom=1mm,
	before skip=15pt,after skip=20pt,
	% borderline west={1.5pt}{0pt}{black,dotted}
	% borderline vertical={0.5pt}{0pt}{black,arrows = {Circle[scale=0.7]-Circle[scale=0.7]}}
	borderline vertical={0.5pt}{0pt}{black}
	% borderline vertical={0.5pt}{0pt}{black},
	% borderline north={0.5pt}{0pt}{white,arrows={Circle[black,scale=0.7]-Circle[black,scale=0.7]}}
	}
    
%---------------------
 

%----------
%ハイパーリンク
% 「%」は以降の内容を「改行コードも含めて」無視するコマンド
\usepackage[%
%  dvipdfmx,% 欧文ではコメントアウトする
luatex,%
pdfencoding=auto,%
 setpagesize=false,%
 bookmarks=true,%
 bookmarksdepth=tocdepth,%
 bookmarksnumbered=true,%
 colorlinks=false,%
 pdftitle={},%
 pdfsubject={},%
 pdfauthor={},%
 pdfkeywords={}%
]{hyperref}
%------------


%参照 参照するときに自動で環境名を含んで参照する
\usepackage[nameinlink]{cleveref}
\let\normalref\ref
\renewcommand{\ref}{\cref}
\crefname{definition}{定義}{定義}
\crefname{proposition}{命題}{命題}
\crefname{theorem}{定理}{定理}
\crefname{lemma}{補題}{補題}
\crefname{corollary}{系}{系}
\crefname{example}{例}{例}
\crefname{practice}{演習問題}{演習問題}
\crefname{equation}{式}{式} 
\crefname{chapter}{第}{第}
\creflabelformat{chapter}{#2#1章#3}
\crefname{section}{第}{第}
\creflabelformat{section}{#2#1節#3}
\crefname{subsection}{第}{第}
\creflabelformat{subsection}{#2#1小節#3}
%----------

%---------------------
%章跨ぎの参照が不具合を起こすための代わり
% \mylabl でラベル付け
\newcommand{\mylabel}[1]{
\label{#1}
\hypertarget{#1}{}
}
% \myref で環境名付きリンクをつける
\newcommand{\myref}[1]{
\hyperlink{#1}{\cref*{#1}}
}
%-----------------

\usepackage{autonum} %参照した数式にだけ番号を振る
% \usepackage{docmute} %ファイル分割
% \begin{document}

% %---------- 1章1節
% \input 1-01.tex
% %---------- 1章2節
% \input 1-02.tex
% % ---------- 1章3節
% \input 1-03.tex
% % ---------- 1章4節
% \input 1-04.tex
% % ---------- 1章5節
% \input 1-05.tex
% % ---------- 1章6節
% \input 1-06.tex
% %---------- 1章7節
% \input 1-07.tex
% % ---------- 1章8節
% \input 1-08.tex
% % ---------- 1章9節
% \input 1-09.tex
% % ---------- 1章10節
% \input 1-10.tex
% % ---------- 1章11節
% \input 1-11.tex
% % ---------- 1章12節
% \input 1-12.tex
% % ---------- 参考文献
% \input reference.tex
% \end{document}
% ----------



\usepackage{bxtexlogo}
\usepackage{amsthm}
\usepackage{amsmath}
\usepackage{bbm} %小文字の黒板文字
\usepackage{physics}
\usepackage{amsfonts}
\usepackage{graphicx}
\usepackage{mathtools}
\usepackage{enumitem}
\usepackage[margin=20truemm]{geometry}
\usepackage{textcomp}
\usepackage{bm}
\usepackage{mathrsfs}
\usepackage{latexsym}
\usepackage{amssymb}
\usepackage{algorithmic}
\usepackage{algorithm}
\usepackage{tikz}
\usepackage{wrapfig}
\usetikzlibrary{arrows.meta}
\usetikzlibrary{math,matrix,backgrounds}
\usetikzlibrary{angles}
\usetikzlibrary{calc}


%----------
%日本語フォント
% \usepackage[deluxe]{otf} platex用 lualatexでは動かない

%----------
%欧文フォント
\usepackage[T1]{fontenc}

%----------
%文字色
\usepackage{color}

%----------
\setlength{\parindent}{2\zw} %インデントの設定

%----------
% %参照した数式にだけ番号を振る cleverrefと併用するとうまくいかない
% \mathtoolsset{showonlyrefs=true}
%----------

%----------
%集合の中線
\newcommand{\relmiddle}[1]{\mathrel{}\middle#1\mathrel{}}
% \middle| の代わりに \relmiddle| を付ける
\newcommand{\sgn}{\mathop{\mathrm{sgn}}} %置換sgn
\newcommand{\Int}{\mathop{\mathrm{Int}}} %位相空間の内部Int
\newcommand{\Ext}{\mathop{\mathrm{Ext}}} %位相空間の外部Ext
\newcommand{\Cl}{\mathop{\mathrm{Cl}}} %位相空間の閉包Cl
\newcommand{\supp}{\mathop{\mathrm{supp}}} %関数の台supp
\newcommand{\restrict}[2]{\left. #1 \right \vert_{#2}}%関数の制限 \restrict{f}{A} = f|_A
\newcommand{\Span}{\mathop{\mathrm{Span}}}
\newcommand{\Ker}{\mathop{\mathrm{Ker}}}
\newcommand{\Coker}{\mathop{\mathrm{Coker}}}
\newcommand{\coker}{\mathop{\mathrm{coker}}}
\newcommand{\Coim}{\mathop{\mathrm{Coim}}}
\newcommand{\coim}{\mathop{\mathrm{coim}}}
\newcommand{\id}{\mathop{\mathrm{id}}}
\newcommand{\Gal}{\mathop{\mathrm{Gal}}}
\renewcommand{\Im}{\mathop{\mathrm{Im}}}
\renewcommand{\Re}{\mathop{\mathrm{Re}}}


\newtheorem{definition}{定義}[section]

\usepackage{aliascnt}

% \newaliastheorem{(環境とカウンターの名前)}{(元となるカウンターの名前)}{(表示される文字列)}
\newcommand*{\newaliastheorem}[3]{%
  \newaliascnt{#1}{#2}%
  \newtheorem{#1}[#1]{#3}%
  \aliascntresetthe{#1}%
  \expandafter\newcommand\csname #1autorefname\endcsname{#3}%
}
\newaliastheorem{proposition}{definition}{命題} 
\newaliastheorem{theorem}{definition}{定理}
\newaliastheorem{lemma}{definition}{補題}
\newaliastheorem{corollary}{definition}{系}
\newaliastheorem{example}{definition}{例}
\newaliastheorem{practice}{definition}{演習問題}

\newtheorem*{longproof}{証明}
\newtheorem*{answer}{解答}
\newtheorem*{supplement}{補足}
\newtheorem*{remark}{注意}
%----------

%----------
%古い記法を注意するパッケージ
\RequirePackage[l2tabu, orthodox]{nag}
%----------


% 定理環境(tcolorbox)
\usepackage{tcolorbox} %箱
\tcbuselibrary{breakable,skins,theorems}
\tcolorboxenvironment{definition}{
	blanker,breakable,
	left=3mm,right=3mm,
	top=2mm,bottom=2mm,
	before skip=15pt,after skip=20pt,
	borderline ={0.5pt}{0pt}{black}
}
\newtcolorbox{emptydefinition}{
	blanker,breakable,
	left=3mm,right=3mm,
	top=2mm,bottom=2mm,
	before skip=15pt,after skip=20pt,
	borderline ={0.5pt}{0pt}{black}
}
%----------
\tcolorboxenvironment{proposition}{
	blanker,breakable,
	left=3mm,right=3mm,
	top=3mm,bottom=3mm,
	before skip=15pt,after skip=15pt,
	borderline={0.5pt}{0pt}{black}
}
\newtcolorbox{emptyproposition}{
	blanker,breakable,
	left=3mm,right=3mm,
	top=3mm,bottom=3mm,
	before skip=15pt,after skip=15pt,
	borderline={0.5pt}{0pt}{black}
}
%----------
\tcolorboxenvironment{theorem}{
	blanker,breakable,
	left=3mm,right=3mm,
	top=3mm,bottom=3mm,
    sharp corners,boxrule=0.6pt,
	before skip=15pt,after skip=15pt,
	borderline={0.5pt}{0pt}{black},
    borderline={0.5pt}{1.5pt}{black}
}
\newtcolorbox{emptytheorem}{
	blanker,breakable,
	left=3mm,right=3mm,
	top=3mm,bottom=3mm,
    sharp corners,boxrule=0.6pt,
	before skip=15pt,after skip=15pt,
	borderline={0.5pt}{0pt}{black},
    borderline={0.5pt}{1.5pt}{black}
}
%----------
\tcolorboxenvironment{lemma}{
	blanker,breakable,
	left=3mm,right=3mm,
	top=3mm,bottom=3mm,
	before skip=15pt,after skip=15pt,
	borderline={0.5pt}{0pt}{black}
}
%----------
\tcolorboxenvironment{corollary}{
	blanker,breakable,
	left=3mm,right=3mm,
	top=3mm,bottom=3mm,
	before skip=15pt,after skip=15pt,
	borderline={1.0pt}{0pt}{black,dotted}
}
\newtcolorbox{emptycorollary}{
	blanker,breakable,
	left=3mm,right=3mm,
	top=3mm,bottom=3mm,
	before skip=15pt,after skip=15pt,
	borderline={1.0pt}{0pt}{black,dotted}
}
%----------
\tcolorboxenvironment{example}{
	blanker,breakable,
	left=3mm,right=3mm,
	top=3mm,bottom=3mm,
	before skip=15pt,after skip=15pt,
	borderline={0.5pt}{0pt}{black}
}
%----------
\tcolorboxenvironment{practice}{
	blanker,breakable,
	left=3mm,right=3mm,
	top=3mm,bottom=3mm,
	before skip=15pt,after skip=15pt,
	borderline={0.5pt}{0pt}{black}
}
%----------
\tcolorboxenvironment{proof}{
	blanker,breakable,
	left=3mm,right=3mm,
	top=2mm,bottom=2mm,
	before skip=15pt,after skip=20pt,
	% borderline west={1.5pt}{0pt}{black,dotted}
	borderline vertical={1pt}{0pt}{black,dotted}
	% borderline vertical={0.8pt}{0pt}{black,dotted,arrows={Square[scale=0.5]-Square[scale=0.5]}}
	}
%----------
\tcolorboxenvironment{supplement}{
	blanker,breakable,
	left=3mm,right=3mm,
	top=2mm,bottom=2mm,
	before skip=15pt,after skip=20pt,
	% borderline west={1.5pt}{0pt}{black,dotted}
	% borderline vertical={0.5pt}{0pt}{black,arrows = {Circle[scale=0.7]-Circle[scale=0.7]}}
	borderline vertical={0.5pt}{0pt}{black}
	% borderline vertical={0.5pt}{0pt}{black},
	% borderline north={0.5pt}{0pt}{white,arrows={Circle[black,scale=0.7]-Circle[black,scale=0.7]}}
	}
%----------
\tcolorboxenvironment{remark}{
	blanker,breakable,
	left=3mm,right=3mm,
	top=1mm,bottom=1mm,
	before skip=15pt,after skip=20pt,
	% borderline west={1.5pt}{0pt}{black,dotted}
	% borderline vertical={0.5pt}{0pt}{black,arrows = {Circle[scale=0.7]-Circle[scale=0.7]}}
	borderline vertical={0.5pt}{0pt}{black}
	% borderline vertical={0.5pt}{0pt}{black},
	% borderline north={0.5pt}{0pt}{white,arrows={Circle[black,scale=0.7]-Circle[black,scale=0.7]}}
	}
    
%---------------------
 

%----------
%ハイパーリンク
% 「%」は以降の内容を「改行コードも含めて」無視するコマンド
\usepackage[%
%  dvipdfmx,% 欧文ではコメントアウトする
luatex,%
pdfencoding=auto,%
 setpagesize=false,%
 bookmarks=true,%
 bookmarksdepth=tocdepth,%
 bookmarksnumbered=true,%
 colorlinks=false,%
 pdftitle={},%
 pdfsubject={},%
 pdfauthor={},%
 pdfkeywords={}%
]{hyperref}
%------------


%参照 参照するときに自動で環境名を含んで参照する
\usepackage[nameinlink]{cleveref}
\let\normalref\ref
\renewcommand{\ref}{\cref}
\crefname{definition}{定義}{定義}
\crefname{proposition}{命題}{命題}
\crefname{theorem}{定理}{定理}
\crefname{lemma}{補題}{補題}
\crefname{corollary}{系}{系}
\crefname{example}{例}{例}
\crefname{practice}{演習問題}{演習問題}
\crefname{equation}{式}{式} 
\crefname{chapter}{第}{第}
\creflabelformat{chapter}{#2#1章#3}
\crefname{section}{第}{第}
\creflabelformat{section}{#2#1節#3}
\crefname{subsection}{第}{第}
\creflabelformat{subsection}{#2#1小節#3}
%----------

%---------------------
%章跨ぎの参照が不具合を起こすための代わり
% \mylabl でラベル付け
\newcommand{\mylabel}[1]{
\label{#1}
\hypertarget{#1}{}
}
% \myref で環境名付きリンクをつける
\newcommand{\myref}[1]{
\hyperlink{#1}{\cref*{#1}}
}
%-----------------

\usepackage{autonum} %参照した数式にだけ番号を振る
% \usepackage{docmute} %ファイル分割
% \begin{document}

% %---------- 1章1節
% \input 1-01.tex
% %---------- 1章2節
% \input 1-02.tex
% % ---------- 1章3節
% \input 1-03.tex
% % ---------- 1章4節
% \input 1-04.tex
% % ---------- 1章5節
% \input 1-05.tex
% % ---------- 1章6節
% \input 1-06.tex
% %---------- 1章7節
% \input 1-07.tex
% % ---------- 1章8節
% \input 1-08.tex
% % ---------- 1章9節
% \input 1-09.tex
% % ---------- 1章10節
% \input 1-10.tex
% % ---------- 1章11節
% \input 1-11.tex
% % ---------- 1章12節
% \input 1-12.tex
% % ---------- 参考文献
% \input reference.tex
% \end{document}
% ----------



\usepackage{bxtexlogo}
\usepackage{amsthm}
\usepackage{amsmath}
\usepackage{bbm} %小文字の黒板文字
\usepackage{physics}
\usepackage{amsfonts}
\usepackage{graphicx}
\usepackage{mathtools}
\usepackage{enumitem}
\usepackage[margin=20truemm]{geometry}
\usepackage{textcomp}
\usepackage{bm}
\usepackage{mathrsfs}
\usepackage{latexsym}
\usepackage{amssymb}
\usepackage{algorithmic}
\usepackage{algorithm}
\usepackage{tikz}
\usepackage{wrapfig}
\usetikzlibrary{arrows.meta}
\usetikzlibrary{math,matrix,backgrounds}
\usetikzlibrary{angles}
\usetikzlibrary{calc}


%----------
%日本語フォント
% \usepackage[deluxe]{otf} platex用 lualatexでは動かない

%----------
%欧文フォント
\usepackage[T1]{fontenc}

%----------
%文字色
\usepackage{color}

%----------
\setlength{\parindent}{2\zw} %インデントの設定

%----------
% %参照した数式にだけ番号を振る cleverrefと併用するとうまくいかない
% \mathtoolsset{showonlyrefs=true}
%----------

%----------
%集合の中線
\newcommand{\relmiddle}[1]{\mathrel{}\middle#1\mathrel{}}
% \middle| の代わりに \relmiddle| を付ける
\newcommand{\sgn}{\mathop{\mathrm{sgn}}} %置換sgn
\newcommand{\Int}{\mathop{\mathrm{Int}}} %位相空間の内部Int
\newcommand{\Ext}{\mathop{\mathrm{Ext}}} %位相空間の外部Ext
\newcommand{\Cl}{\mathop{\mathrm{Cl}}} %位相空間の閉包Cl
\newcommand{\supp}{\mathop{\mathrm{supp}}} %関数の台supp
\newcommand{\restrict}[2]{\left. #1 \right \vert_{#2}}%関数の制限 \restrict{f}{A} = f|_A
\newcommand{\Span}{\mathop{\mathrm{Span}}}
\newcommand{\Ker}{\mathop{\mathrm{Ker}}}
\newcommand{\Coker}{\mathop{\mathrm{Coker}}}
\newcommand{\coker}{\mathop{\mathrm{coker}}}
\newcommand{\Coim}{\mathop{\mathrm{Coim}}}
\newcommand{\coim}{\mathop{\mathrm{coim}}}
\newcommand{\id}{\mathop{\mathrm{id}}}
\newcommand{\Gal}{\mathop{\mathrm{Gal}}}
\renewcommand{\Im}{\mathop{\mathrm{Im}}}
\renewcommand{\Re}{\mathop{\mathrm{Re}}}


\newtheorem{definition}{定義}[section]

\usepackage{aliascnt}

% \newaliastheorem{(環境とカウンターの名前)}{(元となるカウンターの名前)}{(表示される文字列)}
\newcommand*{\newaliastheorem}[3]{%
  \newaliascnt{#1}{#2}%
  \newtheorem{#1}[#1]{#3}%
  \aliascntresetthe{#1}%
  \expandafter\newcommand\csname #1autorefname\endcsname{#3}%
}
\newaliastheorem{proposition}{definition}{命題} 
\newaliastheorem{theorem}{definition}{定理}
\newaliastheorem{lemma}{definition}{補題}
\newaliastheorem{corollary}{definition}{系}
\newaliastheorem{example}{definition}{例}
\newaliastheorem{practice}{definition}{演習問題}

\newtheorem*{longproof}{証明}
\newtheorem*{answer}{解答}
\newtheorem*{supplement}{補足}
\newtheorem*{remark}{注意}
%----------

%----------
%古い記法を注意するパッケージ
\RequirePackage[l2tabu, orthodox]{nag}
%----------


% 定理環境(tcolorbox)
\usepackage{tcolorbox} %箱
\tcbuselibrary{breakable,skins,theorems}
\tcolorboxenvironment{definition}{
	blanker,breakable,
	left=3mm,right=3mm,
	top=2mm,bottom=2mm,
	before skip=15pt,after skip=20pt,
	borderline ={0.5pt}{0pt}{black}
}
\newtcolorbox{emptydefinition}{
	blanker,breakable,
	left=3mm,right=3mm,
	top=2mm,bottom=2mm,
	before skip=15pt,after skip=20pt,
	borderline ={0.5pt}{0pt}{black}
}
%----------
\tcolorboxenvironment{proposition}{
	blanker,breakable,
	left=3mm,right=3mm,
	top=3mm,bottom=3mm,
	before skip=15pt,after skip=15pt,
	borderline={0.5pt}{0pt}{black}
}
\newtcolorbox{emptyproposition}{
	blanker,breakable,
	left=3mm,right=3mm,
	top=3mm,bottom=3mm,
	before skip=15pt,after skip=15pt,
	borderline={0.5pt}{0pt}{black}
}
%----------
\tcolorboxenvironment{theorem}{
	blanker,breakable,
	left=3mm,right=3mm,
	top=3mm,bottom=3mm,
    sharp corners,boxrule=0.6pt,
	before skip=15pt,after skip=15pt,
	borderline={0.5pt}{0pt}{black},
    borderline={0.5pt}{1.5pt}{black}
}
\newtcolorbox{emptytheorem}{
	blanker,breakable,
	left=3mm,right=3mm,
	top=3mm,bottom=3mm,
    sharp corners,boxrule=0.6pt,
	before skip=15pt,after skip=15pt,
	borderline={0.5pt}{0pt}{black},
    borderline={0.5pt}{1.5pt}{black}
}
%----------
\tcolorboxenvironment{lemma}{
	blanker,breakable,
	left=3mm,right=3mm,
	top=3mm,bottom=3mm,
	before skip=15pt,after skip=15pt,
	borderline={0.5pt}{0pt}{black}
}
%----------
\tcolorboxenvironment{corollary}{
	blanker,breakable,
	left=3mm,right=3mm,
	top=3mm,bottom=3mm,
	before skip=15pt,after skip=15pt,
	borderline={1.0pt}{0pt}{black,dotted}
}
\newtcolorbox{emptycorollary}{
	blanker,breakable,
	left=3mm,right=3mm,
	top=3mm,bottom=3mm,
	before skip=15pt,after skip=15pt,
	borderline={1.0pt}{0pt}{black,dotted}
}
%----------
\tcolorboxenvironment{example}{
	blanker,breakable,
	left=3mm,right=3mm,
	top=3mm,bottom=3mm,
	before skip=15pt,after skip=15pt,
	borderline={0.5pt}{0pt}{black}
}
%----------
\tcolorboxenvironment{practice}{
	blanker,breakable,
	left=3mm,right=3mm,
	top=3mm,bottom=3mm,
	before skip=15pt,after skip=15pt,
	borderline={0.5pt}{0pt}{black}
}
%----------
\tcolorboxenvironment{proof}{
	blanker,breakable,
	left=3mm,right=3mm,
	top=2mm,bottom=2mm,
	before skip=15pt,after skip=20pt,
	% borderline west={1.5pt}{0pt}{black,dotted}
	borderline vertical={1pt}{0pt}{black,dotted}
	% borderline vertical={0.8pt}{0pt}{black,dotted,arrows={Square[scale=0.5]-Square[scale=0.5]}}
	}
%----------
\tcolorboxenvironment{supplement}{
	blanker,breakable,
	left=3mm,right=3mm,
	top=2mm,bottom=2mm,
	before skip=15pt,after skip=20pt,
	% borderline west={1.5pt}{0pt}{black,dotted}
	% borderline vertical={0.5pt}{0pt}{black,arrows = {Circle[scale=0.7]-Circle[scale=0.7]}}
	borderline vertical={0.5pt}{0pt}{black}
	% borderline vertical={0.5pt}{0pt}{black},
	% borderline north={0.5pt}{0pt}{white,arrows={Circle[black,scale=0.7]-Circle[black,scale=0.7]}}
	}
%----------
\tcolorboxenvironment{remark}{
	blanker,breakable,
	left=3mm,right=3mm,
	top=1mm,bottom=1mm,
	before skip=15pt,after skip=20pt,
	% borderline west={1.5pt}{0pt}{black,dotted}
	% borderline vertical={0.5pt}{0pt}{black,arrows = {Circle[scale=0.7]-Circle[scale=0.7]}}
	borderline vertical={0.5pt}{0pt}{black}
	% borderline vertical={0.5pt}{0pt}{black},
	% borderline north={0.5pt}{0pt}{white,arrows={Circle[black,scale=0.7]-Circle[black,scale=0.7]}}
	}
    
%---------------------
 

%----------
%ハイパーリンク
% 「%」は以降の内容を「改行コードも含めて」無視するコマンド
\usepackage[%
%  dvipdfmx,% 欧文ではコメントアウトする
luatex,%
pdfencoding=auto,%
 setpagesize=false,%
 bookmarks=true,%
 bookmarksdepth=tocdepth,%
 bookmarksnumbered=true,%
 colorlinks=false,%
 pdftitle={},%
 pdfsubject={},%
 pdfauthor={},%
 pdfkeywords={}%
]{hyperref}
%------------


%参照 参照するときに自動で環境名を含んで参照する
\usepackage[nameinlink]{cleveref}
\let\normalref\ref
\renewcommand{\ref}{\cref}
\crefname{definition}{定義}{定義}
\crefname{proposition}{命題}{命題}
\crefname{theorem}{定理}{定理}
\crefname{lemma}{補題}{補題}
\crefname{corollary}{系}{系}
\crefname{example}{例}{例}
\crefname{practice}{演習問題}{演習問題}
\crefname{equation}{式}{式} 
\crefname{chapter}{第}{第}
\creflabelformat{chapter}{#2#1章#3}
\crefname{section}{第}{第}
\creflabelformat{section}{#2#1節#3}
\crefname{subsection}{第}{第}
\creflabelformat{subsection}{#2#1小節#3}
%----------

%---------------------
%章跨ぎの参照が不具合を起こすための代わり
% \mylabl でラベル付け
\newcommand{\mylabel}[1]{
\label{#1}
\hypertarget{#1}{}
}
% \myref で環境名付きリンクをつける
\newcommand{\myref}[1]{
\hyperlink{#1}{\cref*{#1}}
}
%-----------------

\usepackage{autonum} %参照した数式にだけ番号を振る
\usepackage{docmute} %ファイル分割
\begin{document}

%\chapter{章のタイトル}
\section{H28数学選択}
\fbox{A}
(1)$\varphi(t^2-ut+v)=x^2-(x+y)x+xy=0$である.
よって$\ker \varphi\supset (t^2-ut+v)$である.
$f\in \ker \varphi$とすると,$f(u,v,t)=(t^2-ut+v)g(u,v,t)+h_1(u,v)t+h_2(u,v)$である.
よって$h_1(x+y,xy)x+h_2(x+y,xy)=0$である.$h_1(x+y,xy)x$は対称式の項をもたず,$h_2(x+y,xy)$は対称式の項のみからなるので,$h_1=0,h_2=0$である.

(2)$\varphi(I_{(a,b)})$が素イデアル.$\Leftrightarrow$ $\mathbb{R}[x,y]/\varphi(I_{(a,b)})$が整域.
ここで$\mathbb{R}[x,y]\cong \mathbb{R}[u,v,t]/\ker \varphi$であるから,
$\mathbb{R}[x,y]/\varphi(I_{(a,b)})\cong (\mathbb{R}[u,v,t]/\ker \varphi)/((\ker \varphi,I_{(a,b)})/\ker \varphi)\cong \mathbb{R}[u,v,t]/(\ker \varphi,I_{(a,b)})$である.

$(\ker \varphi,I_{(a,b)})=(t^2-ut+v,u-a,v-b)=(t^2-at+b,u-a,v-b)$であるから,$\mathbb{R}[u,v,t]/(\ker \varphi,I_{(a,b)})\cong \mathbb{R}[t]/(t^2-at+b)$である.
これが整域となるのは$t^2-at+b$が既約であるときである.
すなわち$a^2-4b<0$のときである.

以上より,$\varphi(I_{(a,b)})$が素イデアルであるのは,$a^2-4b<0$のときである.

\fbox{B}
(1)$F=\mathbb{Q}(\alpha)(\beta)$とかける.$\beta$の$\mathbb{Q}(\alpha)$上の最小多項式は$P(X)/(X-\alpha)$の因数であるから$3$次以下である.よって$[F:\mathbb{Q}(\alpha)]\leq 3$である.

(2)$[\mathbb{Q}(\alpha):\mathbb{Q}]=4$より$[F:\mathbb{Q}]=[F:\mathbb{Q}(\alpha)][\mathbb{Q}(\alpha):\mathbb{Q}]=4[F:\mathbb{Q}(\alpha)]$であるから(1)より$[F:\mathbb{Q}]=4,8,12$のいずれかである.

(3)$F/\mathbb{Q}$がGalois拡大であるからGalois群は$(\mathbb{Z}/2\mathbb{Z})^2$か$\mathbb{Z}/4 \mathbb{Z}$の何れかである.
どちらも真部分群を含むから,$F/\mathbb{Q}$は非自明な中間体をもつ.

(4)$P(X)=X^4-4X^2+1$とする.$P(X+1)=X^4+4X^3+2X^2-4X-2$は既約であるから$P(X)$は既約多項式である.
$P(X)$の根は$\pm{\sqrt{2\pm \sqrt{3}}}$である.
$\sqrt{2+\sqrt{3}}\sqrt{2-\sqrt{3}}=1$より$F=\mathbb{Q}(\sqrt{2+\sqrt{3}})$とすれば$F/\mathbb{Q}$はGalois拡大で$[F:\mathbb{Q}]=4$である.

$\sigma(\sqrt{2+\sqrt{3}})=\sqrt{2-\sqrt{3}}$とすると,$\sigma^2(\sqrt{2+\sqrt{3}})=\sigma(\sqrt{2-\sqrt{3}})=\sigma(1/\sqrt{2+\sqrt{3}})=\sqrt{2+\sqrt{3}}$であるから,$\sigma^2=\id$である.

また$\tau(\sqrt{2+\sqrt{3}})=-\sqrt{2+\sqrt{3}}$についても$\tau^2=\id$であるから,$\Gal(F/\mathbb{Q})$は位数$2$の元を$2$つ以上もつ.よって$\Gal(F/\mathbb{Q})\cong (\mathbb{Z}/2\mathbb{Z})^2$より巡回群でない.

(5)$\beta$の$\mathbb{Q}(\alpha)$上の最小多項式の次数は$2$である.したがって$\gamma,\delta$のいずれか一方のみが$\mathbb{Q}(\alpha)$に属す.
$\gamma\in \mathbb{Q}(\alpha)$として一般性を失わない.
よって$\sigma(\alpha)=\gamma$とすれば$\sigma$は$\mathbb{Q}(\alpha)$の恒等写像でない同型射である.

$\mathbb{Q}(\alpha)$の恒等写像でない同型射について,$\sigma(\alpha)$は$\alpha$の$\mathbb{Q}$上共役であるから存在すれば$\sigma(\alpha)=\gamma$のみである.よって一意性も示された.



\end{document}